\section{Challenges}
\label{sec:Challenges}

\begin{description}
	\item[Capability Discovery] Providing the ability to drive interconnected devices assumes the capacity of automatically discovering devices' capabilities in a standardised and uniform way. Similar processes happened for other technologies: for example, an \textsc{Usb} device plugged into a computer automatically exposes its nature (e.g., a pointing or video device) and capabilities. This also requires a large ontology that classifies the many dimensions IoT devices are capable of (e.g., \cite{}).
	
	\item[Low-Level Event Reification] Furthermore, the events selected by the device constructors for operating one device could be radically different from what an end-user would expect to drive a particular solution: e.g., a heart rate monitor needs to scan pulsations every millisecond while what interests the user is the rate per minutes. An effective solution for defining high-level customised usage scenarios requires that a mapping between events perceived within a \textsc{Dsl} are bidirectionally mapped to low-level device events, or other artefacts that govern devices' functionalities. Symetrically, the open question of exposing low-level events to users can be useful for some scenarios, but critical in other situations. 
	
	\item[Protocol Interoperability] A domotic solution with heterogeneous devices would often integrate devices from various constructors, thus communicating through multiple communication protocols. In order to make them communicate efficiently, without forcing end-users to stick with one constructor that can dictate costs and restrictions without any control, a powerful \textsc{Dsl} should provide ways for interoperability over multiple communication protocols, without forcing end-users to understand the protocols' intricacies, version evolution, and restrictions.
	
	\item[Reactive Framework for \textsc{Bl}] 
	
	\item[Complex Event Processing (\textsc{Cep})] consists of deriving meaningful conclusions from a stream of events occuring within a system and of responding to them as quickly as possible. 
	%, by tracking and analysing them with various techniques, such as pattern detection, abstraction, filtering, relationship detection, and transformation.
	
	\item[Decentralisation Features] 
	
	\item[Non-Functional Properties] 
\end{description}
