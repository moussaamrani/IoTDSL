\section{Challenges}
\label{sec:Challenges}

\begin{description}
	\item[Capability Discovery] Providing the ability to drive interconnected devices assumes the capacity of automatically discovering devices' capabilities in a standardised and uniform way. Similar processes happened for other technologies: for example, an \textsc{Usb} device plugged into a computer automatically exposes its nature (e.g., a pointing or video device) and capabilities. Classifying such capabilities could be useful to build an ontology of normalised capabilities that could result in powerful software \textsc{Api}s for manipulating devices. 
		
	\item[Complex Event Processing (\textsc{Cep})] Letting end-users manipulate devices through their low-level capability interfaces could lead to confusion and accidental complexity for defining usage scenarios. Rather, providing a way of reifying low-level internal computations inside devices into high-level events could help end-users leverage the complexity of devices networks and pave the way to freely and transparently manipulate them. Since \textsc{Cep} consists of deriving meaningful conclusions from a stream of events occurring within a system and of responding to them as quickly as possible, it provides a solution for extracting meaningful events from low-level computations. However, for a solution to be complete and useful, the reverse direction should be addressed: high-level actions should be adequately translated into low-level devices' actuations. 
		
	\item[Protocol Interoperability] A domotic solution with heterogeneous devices would often integrate devices from various constructors, thus communicating through multiple communication protocols. In order to make them communicate efficiently, without forcing end-users to stick with one constructor that can dictate costs and restrictions without any control, a powerful \textsc{Dsl} should provide ways for interoperability over multiple communication protocols, without forcing end-users to understand the protocols' intricacies, version evolution, and restrictions.
	
	\item[Non-Functional Properties] A powerful \textsc{Dsl} should encompass typical non-functional properties of device networks to ensure long-life and secure realisation of scenarios. \emph{Performance} is crucial, and depends both on the devices capabilities but also on the quality of the network communications: any source of latency could have a dramatic impact that can lead to critical situations. \emph{Resource availability}, both in terms of computation and memory capability, but also in terms of energy is another crucial bottleneck for the adoption of \textsc{Dsl}s as a solution for defining scenarios: the code generated from the \textsc{Dsl} should not overload the devices with repetitive communications or unnecessary computations that would drain the device battery. \emph{Security} is yet another concern with respect to two aspects. First, sensitive data could be exposed through the communication network, endangering users privacy. Second, functionalities could be locked to be used only by authorised users.

\end{description}
