\documentclass[a4paper,twoside]{article}

\usepackage{amsfonts, amssymb, mathabx}
\usepackage{boxedminipage}
\usepackage{multirow, textcomp}
\usepackage{color, xcolor}
\usepackage{boxedminipage, xspace, url}

\usepackage{graphics, graphicx, xcolor, setspace}
\graphicspath{{../img/}{model/}}

%%%%%%%%%%%%%%%%%%%%%%%%%%%%%%%%%%%%%%%%%%%%%%%%%%%%%%%%%%%%%%%%%%%%%%%%%%
%%%%%%%%%%%%%%%%%%%%%%%%%%%%%%%%%%%%%%%%%%%%%%%%%%%%%%%%%%%%%%%%%%%%%%%%%%
%%%%%%%%%%%%%%%   TODO & COMMENTS

\usepackage{todonotes}

\newif\ifdraft\drafttrue
\ifdraft
   \newcommand\todos[1]{\medskip\todo[inline]{TODO (all): #1}}
   \newcommand\moussa[1]{\medskip\todo[color=green!40,inline]{TODO (Moussa): #1}}
   \newcommand\nicolas[1]{\medskip\todo[color=purple!40,inline]{TODO (Nicolas): #1}}
   \newcommand\xavier[1]{\medskip\todo[color=blue!40,inline]{TODO (Xavier): #1}}
\else
   \newcommand\todos[1]{}
   \newcommand\moussa[1]{}
   \newcommand\nicolas[1]{}
   \newcommand\xavier[1]{}
\fi
%%%%%%%%%%%%%%%%%%%%%%%%%%%%%%%%%%%%%%%%%%%%%%%%%%%%%%%%%%%%%%%%%%%%%%%%%%



\usepackage{xcolor}
\definecolor{myblue}{HTML}{406E86}
\definecolor{codeblue}{HTML}{004385}
\definecolor{codegrey}{HTML}{4D4D4D}
\definecolor{rulerline}{HTML}{A0A2AC}
\definecolor{codegreen}{HTML}{518D71}
\definecolor{codeviolet}{HTML}{7F0055}
\definecolor{darkblue}{HTML}{08088A}
\definecolor{mygrey}{HTML}{4D4D4D}
\usepackage{inconsolata}
\usepackage{listingsutf8}
\usepackage{caption}[2015/09/20]
\usepackage{enumitem}


\usepackage[T1]{fontenc}
\usepackage[utf8]{inputenc}
\usepackage{SCITEPRESS}     % Please add other packages that you may need BEFORE the SCITEPRESS.sty package.

\newcommand{\IOT}{IoT\xspace}
\newcommand{\IOTDSL}{\textsf{IoTD\scriptsize\MakeUppercase{sl}}\xspace}
\newcommand{\DSL}{\textsc{Dsl}\xspace}
\newcommand{\DSLS}{\textsc{Dsl}s\xspace}
\newcommand{\MDE}{\textsc{Mde}\xspace}

%%%%%%%%%%%%%%%%%%%%%%%%%%%%%%%%%%%%%%%%%%%%%%%%%%%%%%%%%%%%%%%%%%%%%%%%%%
%%%%%%%%%%%%%%%%%%%%%%%%%%%%%%%%%%%%%%%%%%%%%%%%%%%%%%%%%%%%%%%%%%%%%%%%%%
%%% IoTDSL Syntax Definition
%%%%%%%%%%%%%%%%%%%%%%%%%%%%%%%%%%%%%%%%%%%%%%%%%%%%%%%%%%%%%%%%%%%%%%%%%%
\lstdefinelanguage{iotdsl}{
  morecomment=[l]{//},
  morecomment=[s]{/*}{*/},
  morestring=[d]",
  morestring=[d]',
  morekeywords={
    configuration, node, from, to, via,
    sensing, actuating, gateway, device,
    rule, when, before, after, and, do, in, out, within
  }
}

\lstset{
  basicstyle=\fontsize{6}{6}\ttfamily,
  basewidth={0.55em,0.45em},
  keywordstyle=\bfseries\color{codeviolet},
  commentstyle=\itshape\color{mygrey},
  stringstyle=\color{darkblue},
  identifierstyle=\color{black},
  numbers=left,
  numberstyle=\tiny\color{rulerline},
  stepnumber=1, 
  numbersep=5pt,
  firstnumber=1, 
  frame=lines,
  rulecolor=\color{rulerline},
  tabsize=2,
  breaklines=true,
  aboveskip=4pt,
  belowskip=4pt,
  showspaces=false,
  showstringspaces=false,
  captionpos=b,
  literate={~} {$\sim$}{1},
  showlines=true
}
%%%%%%%%%%%%%%%%%%%%%%%%%%%%%%%%%%%%%%%%%%%%%%%%%%%%%%%%%%%%%%%%%%%%%%%%%%
%%%%%%%%%%%%%%%%%%%%%%%%%%%%%%%%%%%%%%%%%%%%%%%%%%%%%%%%%%%%%%%%%%%%%%%%%%



\begin{document}

\renewcommand{\thelstlisting}{\arabic{lstlisting}}

\title{Towards Rule-based Semantics for the Internet of Things}
\author{\authorname{Moussa Amrani\sup{1}, Fabian Gilson\sup{1}, Abdelmounaïm Debieche\sup{1}, Vincent Englebert\sup{1}}
\affiliation{\sup{1}University of Namur, Belgium\\
\email{\{Moussa.Amrani, Abdelmounaim.Debieche, Vincent.Englebert, Fabian.Gilson\}@unamur.be}
}}

\keywords{Model-Driven Engineering, Internet of Things, Domain-Specific Language, Rule-based Semantics}

\abstract{Hidden behind the Internet of Things (\IOT), many actors are activelly filling the market with devices and services. From this profusion of actors, a large amount of technologies and APIs, sometimes proprietary, are available, making difficult the interoperability and configuration of systems for \IOT technicians. In order to define and manipulate devices deployed in domestic environments, we propose \IOTDSL, a Domain-Specific Language meant to specify, assemble and describe the behaviour of interconnected devices. Relying on a high-level rule-based language, users in charge of the deployment of \IOT infrastructures are able to describe and combine in a declarative manner structural configurations as well as event-based semantics for devices. This way, language users are freed from technical aspects, playing with high-level representations of devices, while the complexity of the concrete implementation is handle in a dedicated layer where high-level rules are mapped to vendor's API.}

\onecolumn 
\maketitle 
\normalsize 
\vfill


\section{Introduction}
\label{sec:Introduction}

Facing the explosion of available connected devices, many vendors are jumping into the market, proposing a large spectrum of products ranging from connected devices to associated end-user services~\cite{lee-15}. This results in a wide heterogeneity in software and hardware implementations, as well as a list of major concerns and opportunities in terms of interoperability, data management, privacy and scalability~\cite{chaqfeh-12}.

As the \IOT infiltrates many aspects of people's life through their cars, heating systems, phones and so forth, a critical challenge is to provide end-users the possibility to benefit from the plethora of connected devices and configure them for their particular needs. From past research conducted at the University of Namur, we identified how difficult it is to make \textit{things} cooperate, and describe configurations because of the major influence of distinct technologies. In order to hide vendor-specific implementation details, we target a dedicated technology-agnostic environment to adapt and combine \IOT solutions.

Model-Driven Engineering (\MDE) has been recognised during the last decade as a software practice and technologies dedicated to the design, management and evolution of computer languages enabling automatic generation of production code, diverse types of analysis and early verifications. Following this trend, we introduce \IOTDSL, a prototype Domain-Specific Language (\DSL) meant to capture \IOT devices capabilities and their deployment configurations, while providing a declarative way to end end-users, letting them to achieve their own scenarios.

This paper starts by identifying in Section~\ref{sec:Context-Challenges} the \IOT challenges specific to the use of \DSLS and \MDE techniques for providing end-users control over their devices. We introduce in Section~\ref{sec:IoTDSL} \IOTDSL, our prototype \DSL to specify and interconnect devices in an intuitive and general way. We then in Section~\ref{sec:RW} overview the use of \DSLS for \IOT, comparing existing approaches with ours and assessing them against the challenges we identified. Section~\ref{sec:Conclusion} concludes and presents the main lines of work ahead to transform our prototype in a fully functional \DSL.
\section{Context \& Challenges}
\label{sec:Context-Challenges}

\IOT devices could be used in many different situations for many different purposes. Consider these two examplars for smart homes:
\begin{enumerate}
	\item Alice leaves in a house equipped with many devices: a door and window lock detectors at key locations in the house, many devices that control the house's light, temperature and humidity, as well as security devices for detecting smoke and carbon dioxide, and many entertainment devices for TV, music, and sport. She is an active, working person, so she reconfigures her home devices often to accommodate her changing lifestyle: for example during winter, she exercices at home and needs her bathroom and living room warm enough, but not too much; while during summer, she likes walking and running outside, and would like her home to stay safe and to be notified of any intrusion. 
	
	\item Bob is 58 years old and leaves alone at home, while his children have a family life on the other side of the city. Like many elderly, Bob suffers from ageing diseases, but prefers to stay in a familiar environment rather than leaving in an institution. His house is equipped with devices that detects potential falls that can be dramatic for him, and with traditional entertainment devices such as a smart TV and phone. He wants to feel safe at home: if he falls and cannot stand up anymore in the bathroom, he would like to warn his family and notify daycare nurses to help him quickly cope with these situations. 
\end{enumerate}
These typical situations share two commonalities. First, both houses integrate common connected devices fulfilling the same functionalities (e.g., monitoring the temperature, or sending messages after a situation recognised as abnormal is detected), but are likely different in terms of vendors, communication protocols and detailed properties (for example, a temperature sensor could be paired with a heating system to maintain a given temperature inside the house; while in other configurations, both devices are clearly separated). Second, the way these devices interact highly depends on the end user, and more likely for the same user, on the present context: Alice's exemplar explicitly calls for different actions for different periods during the year; while Bob's examplar can easily be replicated in another house, but with different sets of concerns. 

We argue that a good way of capturing those potential variations of devices and situations is to provide to end users (be they home inhabitants like Alice, or technicians in charge of equipping house like Bob's) a \DSL that possess the following components:
\begin{description}
	\item[Device Description] A precise inventory of the devices used in a specific deployment as well as the high-level capabilities of these devices, described in terms that are immediately understandable by end-users, as opposed to conveying technical details about how those devices precisely execute;
	
	\item[Network Description] A way to capture where each device is located and how it is possible to communicate with it, in order to receive or send data to it;

	\item[Dynamics] A way to describe the interactions wished by end-users, i.e. how to leverage the functionalities of the devices to effectively realise one or several scenarios that are convenient for the end-users.  
	
\end{description}
Those components are obviously not sufficient for obtaining a fully-fledged solution that becomes adaptable to any situation, but they still represent the necessary steps for providing end-users the capacity to manipulate a collection of devices directly without resorting on specific technologies.

Nevertheless, the definition of such a \DSL raises many questions directly related to the many hypotheses assumed by those \DSL components.


















\begin{description}
	\item[Capability Discovery] Providing the ability to drive interconnected devices assumes the capacity of automatically discovering devices' capabilities in a standardised and uniform way. Similar processes happened for other technologies: for example, an \textsc{Usb} device plugged into a computer automatically exposes its nature (e.g., a pointing or video device) and capabilities. Classifying such capabilities could be useful to build an ontology of normalised capabilities that could result in powerful software \textsc{Api}s for manipulating devices. 
		
	\item[Complex Event Processing (\textsc{Cep})] Letting end-users manipulate devices through their low-level capability interfaces could lead to confusion and accidental complexity for defining usage scenarios. Rather, providing a way of reifying low-level internal computations inside devices into high-level events could help end-users leverage the complexity of devices networks and pave the way to freely and transparently manipulate them. Since \textsc{Cep} consists of deriving meaningful conclusions from a stream of events occurring within a system and of responding to them as quickly as possible, it provides a solution for extracting meaningful events from low-level computations. However, for a solution to be complete and useful, the reverse direction should be addressed: high-level actions should be adequately translated into low-level devices' actuations. 
		
	\item[Protocol Interoperability] A domotic solution with heterogeneous devices would often integrate devices from various constructors, thus communicating through multiple communication protocols. In order to make them communicate efficiently, without forcing end-users to stick with one constructor that can dictate costs and restrictions without any control, a powerful \textsc{Dsl} should provide ways for interoperability over multiple communication protocols, without forcing end-users to understand the protocols' intricacies, version evolution, and restrictions.
	
	\item[Scalability]
	
	\item[Data Management]  
	
	\item[Non-Functional Properties] A powerful \textsc{Dsl} should encompass typical non-functional properties of device networks to ensure long-life and secure realisation of scenarios. \emph{Performance} is crucial, and depends both on the devices capabilities but also on the quality of the network communications: any source of latency could have a dramatic impact that can lead to critical situations. \emph{Resource availability}, both in terms of computation and memory capability, but also in terms of energy is another crucial bottleneck for the adoption of \textsc{Dsl}s as a solution for defining scenarios: the code generated from the \textsc{Dsl} should not overload the devices with repetitive communications or unnecessary computations that would drain the device battery. \emph{Security} is yet another concern with respect to two aspects. First, sensitive data could be exposed through the communication network, endangering users privacy. Second, functionalities could be locked to be used only by authorised users.

\end{description}


\section{IoTDSL}
\label{sec:IoTDSL}

Based on the challenges identified in section~\ref{sec:Motivation}, we now introduce \IOTDSL, our \DSL devoted to facilitate the high-level manipulation of \IOT systems. At the heart of \IOTDSL are two governing principles. First, we promote a clean separation of concerns for all aspects the \DSL has to handle, by specifying three sub-languages. We believe this approach to be scalable, and to support independent evolutions of each concern without impacting the other aspects, since those aspects are composed through well-defined interfaces. Second, our \DSL relies on events, a natural paradigm for specifying various models of interactions that is widely used in embedded and critical systems, and where a clear separation between the system and its environment is performed, further empowering the separation of concerns. Despite its early stage of development, \IOTDSL shows its ability to capture the definition of small-scale \IOT systems appropriately.

Building a well-calibrated \DSL is known to be difficult and error-prone. It usually requires a broad expertise of the domain under consideration before a consensus emerges on the domain's key concepts and how to effectively represent them. Fortunately, \MDE technologies operated substantial breakthrough over the past decade, allowing language designers to define their own \DSL structures and user interfaces more easily. Adopting such a trend, we have built an early prototype for our \DSL under \textit{GeMoC} \cite{bousse-16}, a \MDE framework that supports both visual and textual representations as concrete syntaxes and maintains a full synchronisation between them. Since we are at early development stage, only a textual syntax is currently available to modellers, but other syntaxes, even graphical ones, can be added smoothly thanks to \textit{GeMoc}.

To illustrate our proposal, we start by depicting the language's metamodel where we highlight its main features. We then illustrate the available constructs in \IOTDSL with the hypothetical scenario presented in section~\ref{sec:Motivation}.

\subsection{Type Definition}
\label{sec:IoTDSL-TD}

The first task is to provide a description of which capabilities each device included in the \IOT system possess: how each device may provide information about the environment through a \emph{sensing} operation; and how it could react and influence it through \emph{actuations}. Our framework currently requires that an advanced user that is able to reason properly about how to effectively manipulate a device and extract the relevant information, but is flexible enough to accomodate automation in the future, so that information about devices could be automatically extracted from pre-existing devices databases (either from a knowledge database the \IOT system is connected to, or from a library of \emph{off-the-shelf devices}).

%In this section, we define \IOT devices' types, \textit{i.e.} which capabilities are available to the users in terms of getting information from the environment, \textit{a.k.a.} sensing, and operating on the environment, \textit{a.k.a} actuating. In our framework, type definitions either come from an advanced user who is able to reason properly about a particular device and extract the relevant information, or from a pre-existing devices database, either being a repository the system is connected to, or a library of \textit{devices-off-the-shelf}. 

\begin{figure*}%
  \centering  
  \includegraphics[width=.92\linewidth]{IoTDevice-MM.png}%
  \caption{Metamodel of \IOTDSL, separated in three concerns: \emph{Type Definition} captures devices' capabilities (top green part), \emph{Network Configuration} details how device instances are connected to each others (middle purple part), \emph{Business Rules} defines the functionalities expected from the IoT installation (bottom yellow part).}%
  \label{fig:IoTDevice-MM}%
\end{figure*}

The concepts dedicated to type definition are shown in Figure~\ref{fig:IoTDevice-MM} (green background). This part is similar to the notion of \textsf{Classifier} in \textsc{Mof}-like languages: a \textsf{Type} is either a \textsf{PrimitiveType}, or a user-defined \textsf{DeclaredType}. We distinguish between general \textsf{Gateway}s, which centralise information and processing, and \textsf{Node}s deployed in the environment and communicating with gateways, and which possess capabilities to interact with the environment. A \textsf{Capability} is basically a parameterised event that drives the node to either capture data from the environment, to act on it, or to perform both. This abstract view of an ``thing'' allows us to manipulate any device at a high abstraction level, exhibiting a clean and uniform interface for end-users based on device capabilities. Since \textsf{Node}s are \textsf{Type}s themselves, it remains possible to reference them as a parameter for the purpose of dynamic discovery across devices.

Listing~\ref{lis:RE-TypeDeclarations} illustrates how the devices in Figure~\ref{fig:scenario} are declared in \IOTDSL. Each device is introduced by the keyword \textsf{device}, possesses a name and lists capabilities that correspond to reporting events (\textsf{sensing}) or operating over the environment (\textsf{actuating}). 

\begin{table}
	\begin{minipage}[b]{.45\textwidth }%
		\begin{lstlisting}[language=iotdsl]	
gateway Middleware
device DoorDetector {
	sensing opened()
	sensing closed()
}
device MotionDetector {
	sensing moving()
}
device ToggleSwitch {
	sensing toggle();
}
		\end{lstlisting}
	\end{minipage}\hfill%
	\begin{minipage}[b]{.45\textwidth}
		\begin{lstlisting}[language=iotdsl, firstnumber=12]
		
device LightSensor {
	sensing light()
}
device LightBulb {
	actuating on()
	actuating off()
}	
device Alarm {
	actuating sound()
}
		\end{lstlisting}
	\end{minipage}
	\caption{Type declarations in \IOTDSL: capabilities as high-level events.}
	\label{lis:RE-TypeDeclarations}
\end{table}

Any IoT system should declare a special device, introduced with the keyword \textsf{gateway}, that centralises data from all devices connected to it, as we will show in Section \ref{sec:IoTDSL-NetworkConfiguration}. This device will be responsible of the event orchestration and will host the \CEP engine that embeds the implementation of the business rules. Also note that the above model is the \textit{user-defined} part of \IOTDSL. In the background, abstract events attached to all devices will need to be mapped to concrete low-level \textsc{Api}s events using a dedicated mapping language that is out of the scope of this paper.


\subsection{Network Configuration}
\label{sec:IoTDSL-NetworkConfiguration}

The configuration constructs of \IOTDSL are specified in the purple-part of Figure~\ref{fig:IoTDevice-MM}. Since we use an architecture centralised around gateways, a network \textsf{Configuration} is a graph-like structure where vertices are \textsf{Gateway}s and \textsf{NodeInstance}s (so that instances may communicate with each others), while edges represent \textsf{CommunicationPath}s (or channels). Such paths define, among others, one or more protocols used to interact. We actually rely on existing platforms, such as OpenRemote (\url{http://www.openremote.org}) or SmartThings ({\url{https://www.smartthings.com}) to handle the intricate details of the protocols since such details are, from an end-user point of view, technical aspects rather than essential matters of the configuration itself. By knowing which protocols are used between each pair of devices, we can automatically perform data conversion in the proper format required by the protocols: most of those protocols are already implemented in \textit{General-Purpose Programming Languages} (\textsc{Gpl}s), like Java or C.

Listing~\ref{lis:RE-Network} shows an instantiation as well as the connection that conforms to the types given in Listing~\ref{lis:RE-TypeDeclarations} and the configuration presented in Figure~\ref{fig:scenario}.
	
	
\begin{table}
	\begin{minipage}[b]{.45\textwidth }%
		\begin{lstlisting}[language=iotdsl]	
configuration SmartHouse {
	node middle   		   : Middleware
	node alarm					 : Alarm
	node toggle          : ToggleSwitch
	node frontDoor			 : DoorDetector
	node parentDoor			 : DoorDetector
	node childDoor			 : DoorDetector
	node balconyDoor 		 : DoorDetector
	node outLight				 : LightSensor
	node livingLight		 : LightSensor
	node livingBulb			 : LightBulb
	node bathroomBulb    : LightBulb
	node foyerBulb       : LightBulb
	node balconyMotion	 : MotionDetector
	node foyerMotion  	 : MotionDetector
	node hallMotion	     : MotionDetector
		\end{lstlisting}
	\end{minipage}\hfill%
	\begin{minipage}[b]{.45\textwidth}
		\begin{lstlisting}[language=iotdsl, firstnumber=17]
	from alarm			   to middle via IP
	from frontDoor 	   to middle via IP
	from parentDoor    to middle via IP
	from childDoor     to middle via IP
	from balconyDoor   to middle via IP
	from outLight      to middle via IP
	from livingLight   to middle via IP
	from livingBulb    to middle via IP
	from bathroomBulb  to middle via IP
	from corridorBulb  to middle via IP
	from balconyMotion to middle via IP
	from frontMotion   to middle via IP
	from hallMotion    to middle via IP
}
		\end{lstlisting}
		\vspace*{.3cm}
	\end{minipage}
	\caption{Network Configuration in \IOTDSL for our smart house.}
	\label{lis:RE-Network}
\end{table}
	
A specific device is considered as an instance of a defined type such that particular devices with the same set of capabilities may be distinguished via identifiable unique references. Communications are purely declarative and only mention the protocol type (introduced by the \lstinline[language=iotdsl]{via} keyword). In our example, we simply decided to use an \textsf{IP} protocol for all bindings. Note that a similar mapping process that the one described at the end of Section~\ref{sec:IoTDSL-TD} is required to reify abstract connections between \textsf{NodeInstances} to physical ports and protocols, but again, these mapping statements are outside of the scope of this paper. 


\subsection{Business Rules}
\label{sec:IoTDSL-BusinessRules}

Business rules are the core of the manipulation of \IOT systems and compose the third part of \IOTDSL as detailed in the bottom yellow part of Figure~\ref{fig:IoTDevice-MM}. This last sub-language relies on an event-based framework that allows to specify a set of \textsf{Rule}s expressing the many functionalities an end-user wants to achieve in his/her concrete configuration. 

In \IOTDSL, a rule is identified by the keyword \textsf{rule} followed by an unique identifier and its body is of the form << \texttt{\textbf{\color{codeviolet}{when}} (trigger) \textbf{\color{codeviolet}{do}} \{ reaction \}} >>, where the \textsf{trigger} specifies a boolean condition under which given \textsf{reaction} must be triggered. Rules' \textsf{trigger}s are cyclically evaluated against the surrounding environment and a \textsf{reaction} defines a sequential or parallel combination of capabilities, enabling to sort actions by, or require data from some identifiable devices. Inside the \textsf{trigger}, users typically check events to evaluate their presences, but as we will see in the following examples, they are also able to check on their absence or returned values. A typical \textsf{reaction} may be to switch on all lights in a house, or only the ones of a certain type by sending new events.

For now, our approach is purely middleware-oriented: rules are gathered and evaluated into a single gateway. For efficiency and resource consumption reasons, we are also exploring how to automatically identify parts of the business logic that can be exported to advanced nodes with sufficient processing and power resources in order to lower network and gateway overuses.

We now illustrate how our smart house scenario presented in Section \ref{sec:Motivation} can be translated into business rules in \IOTDSL with the devices' definitions detailed in Listings~\ref{lis:RE-TypeDeclarations} and \ref{lis:RE-Network}. We identified three different situations to depict the usage of business rules and highlight the main assets of \IOTDSL.

\subsubsection*{Light on when coming home}

When Alice gets home (and thus opens the front door), she wants the lights to be automatically switched on in the foyer and in the living room.

\begin{lstlisting}[language=iotdsl,label=lis:home-rule,caption=\IOTDSL business rule to switch on the lights when coming home]
rule SwitchLightsWhenEnter:
  when (frontDoor.opened() and after foyerMotion.moving()) do {
    foyerBulb.on()
    livingBulb.on()
  }
\end{lstlisting}

As we are dealing with concurrent events, we introduce two special operators on top of common boolean operators, called \textsf{before} and \textsf{after}. With these event modifiers we are able to check that two events connected by a binary operator (an \texttt{and} in this particular example) appear in a reasonable time window. In the above rule, the \texttt{frontDoor} must be opened and \textit{quasi subsequently} followed by a movement detection by the \texttt{foyerMotion}.

One or more events, declared in a rather imperative way may be triggered from a rule. The concrete execution semantics is left to the user, where in some cases, all events will be produced asynchronously or simultaneously, where in other they will be produced sequentially. This freedom in the language is meant to cover any execution semantics, especially because this semantics is often domain-specific and depends on technical details our end users probably do not want to deal with at that level of abstraction.

\subsubsection*{Light on when child wakes up at night}
	
When Alice's child wakes up at night, she would like to have the light in the bathroom to be switched on to prevent her from falling or injuring herself. Analogously, she wants the light to be switched off when she gets back to sleep afterwards.

\begin{lstlisting}[language=iotdsl,label=lis:night-rule,caption=\IOTDSL business rules to switch on\//off the lights at night]
rule SwitchBathroomLightOnAtNight:	
  when (outLight.light(l) == false and livingLight.light(l) == false 
  		and (childDoor.opened() and after hallMotion.moving())) do {
  	bathroomBulb.on()
  }
  
rule SwitchBathroomLightOffAtNight:	
	when (not hallMotion.moving() within 3 min from childDoor.closed() 
			and outLight.light(l) == false) do {
		bathroomBulb.off()
	}
\end{lstlisting}

The first rule expresses that, at night, when both \texttt{LightSensor} are sending \texttt{false} values and when Alice's child opens her door \textit{immediately} followed by some movements in the hall, the light in the bathroom is switched on. As the \texttt{LightSensor.light()} events are typed events sending boolean values, we have to evaluate their values with the parameter \texttt{l} implicitly resolved as the event argument. We also note that at line 3, we must use parenthesis because the \textsf{after} modifier needs to know what (set of) events must precede the other(s) event(s) on the right hand side. 

The second rule in Listing~\ref{lis:night-rule} checks on the absence of an event \textsf{within} a given time frame after (\textit{i.e.} \textsf{from}) another event has been detected. In this rule, we state that when no more movement have been detected in the hall for three minutes after the door of the child's room has been closed, the light in the bathroom must be switched off. In \IOTDSL, users are then able to check on the absence of an event within a time frame of their choice (with a set of predefined time units). It is worth noting that the \textsf{not} qualifiers is only used to check on the absence of an event, where the \texttt{outLight.light(l) == false} checks on the return value of the event. We could have had an event returning \textsf{integer} values like \texttt{0} or \texttt{1} so the boolean expression would have been \texttt{outLight.light(l) == 0}.

\subsubsection*{Child on balcony without surveillance}
	
There is a critical situation when Alice's child goes onto the balcony without supervision. To that end, Alice has placed a switch button that an adult has to press when going on the balcony. If the toggle switch is not pressed within 3 sec after someone gets onto the balcony, the alarm must ring.

\begin{lstlisting}[language=iotdsl,label=lis:balcony-rule,caption=\IOTDSL business rules ring the alarm when Alice's child alone on balcony]
rule AlarmWhenChildOnBalcony:	
  when (not toggle.toggle() within 3 sec from 
  		(balconyDoor.opened() and after balconyMotion.moving())) do {
    alarm.sound()
  }
\end{lstlisting}

In other words, the above rule states that after the door of the balcony has been opened and any movement has been detected on the balcony, but no toggle button has been pressed in the next 3 seconds, the alarm must ring. This last rule is somewhat similar to the ones in Listing~\ref{lis:night-rule} except that we combine the absence of an event in a time frame directly followed by the detection of other events in a sequential or in quasi simultaneity. In \IOTDSL, users are then able to fully describe a sequential execution of events, \textit{i.e} a complete business workflow, that must be checked before triggering some \textsf{reaction}.






\section{Related Work}
\label{sec:RW}

A series of overviews have been recently conducted on several aspects of \IOT. In \cite{alfuqaha-15,xu-14a}, the authors reviewed the applications, protocols and technologies used in the distinct \IOT layers, while \cite{singh-14,gubbi-13} focused on architectural aspects and \cite{tan-10,xu-14b} reviewed security ones. Most of these contributions identify a number of challenges crossing the application domain of a \DSL for \IOT, from which we identified the most relevant ones to our contribution in Section~\ref{sec:Motivation} and to which we confront our framework in Section~\ref{sec:Discussion}.

Capturing variations of a domain with explicit constructs close to the domain concepts resides at the essence of \DSLS. In that regard, many \DSLS were proposed for various purposes in the \IOT stack. \textsc{Chariot}~\cite{pradhan-15} addresses Cyber-Physical Systems by providing a component model that clearly distinguishes between communication and computation, while ensuring resilience features in highly reconfigurable systems. In~\cite{brandtzaeg-12} is presented a \DSL aimed at facilitating the deployment of applications, based on a component model of the environment used to locate the architecture nodes where business logic can be leveraged. \textsc{Alph}~\cite{munnelly-08} is a \DSL for ubiquitous healthcare that focuses on three concerns: mobility, by helping users to manage frequent devices disconnections; context-awareness to adapt application behaviour to environmental changes; and infrastructure, for managing the heterogeneity of communication protocols. Midgar~\cite{garcia-14} offers a visual interface to support end-users in controlling interconnected devices and generate the glue application making these devices interoperate. In~\cite{salihbegovic-15}, the authors present a visual \DSL for capturing the features and intercommunications of devices distributed in various application domains spanning from smart homes to patient monitoring. These contributions target different application domains at different abstraction levels, but possess every key features we identified in Section~\ref{sec:Motivation} in a more or less explicit way. Since \IOTDSL targets end-users with no prior knowledge in programming, we contrast with these contributions by offering a more intuitive, declarative style for expressing the system's dynamics through semantics rules that are compilable into a runnable \CEP engine.

ThingML~\cite{harrand-16} is the closest contribution to our \DSL: it uses a similar device description with messages and communication ports attached to devices, but describes the dynamics of devices and systems through state machines, which appear to be more obscure for end-users. However, the conceptual drawbacks are similar in both paradigms: state machines need to be deterministic on their transitions, while rules have to avoid multiple concurrent firing to avoid executing several rules at the same time. 

Other approaches, \textit{e.g.}~\cite{bhandari-13,cheng-16}, relying on the \textit{Event Condition Action} (ECA) paradigm, share a similar view for \IOT devices orchestration through \CEP, though not having the same expressiveness for devices' definition as we propose, especially with time frames and event compositions. In~\cite{shimokura-07}, the authors add \textit{pre-} and \textit{post-conditions} to ECA rules but they still do not address time frames constraints too.

All previous contributions take advantages of \MDE technologies and tools. More general \MDE framework like GeMoC~\cite{bousse-16} or ThingML allow to specialise the description of interconnected devices, for example to describe Arduino systems specifically in ArduinoML~\cite{mosser-14}. On contrary, \IOTDSL framework concentrates on generating executable rules from user-defined requirements.
\section{Conclusion and Future Work}
\label{sec:Conclusion}

The Internet of Things promotes the usage of various interconnected devices that pro\-mise to help end users achieve more automation in daily life recognisable scenarios. As such, a smart home could automatically close doors and switch off lights when the inhabitants leave, or facilitate life routines by assisting in tasks like preparing coffee just in time. The flip side of the coin resides in the ever growing spectrum of connected devices proposed by vendors that spot the market as a good opportunity to make profit, without ensuring a minimal interoperability between their products and those available from other merchants. As a result, the promise seems far from happening in the near future without powerful solutions to catalogue connected devices, to make them exchange relevant data and act in a disciplined way. Furthermore, without bringing end users at the heart of their own story and providing them tools to define, drive and adapt their own scenarios, vendors will always keep a grasp at the \IOT market.

In this paper, we explore a first step towards achieving this large challenge by proposing a prototype that aims at raising end users as main actors of how smart home devices interact for their own needs. Aware of the many challenges surrounding \IOT systems including reusability, interoperability, scalability and non-functional properties, we designed our solution as an evolving and decentralised tool that allows end users to specify their own scenarios based on so-called rule-based definitions. Our prototype takes the form of a Domain-Specific Language (\DSL) associated with a code generator that produces executable code designed to run on a Complex Event Processing (\CEP) engine, as well as emulation code dedicated to simulate the whole system before effectively deploying it with concrete devices.

Our prototype \IOTDSL clearly separates three necessary aspects when describing solutions for \IOT systems. First, it captures devices capabilities as high-level events that are meaningful to end users, thus hiding the intricacies of low-level manipulation of \DSLS into our platform. Second, it describes device interconnections in a declarative language with predefined communication protocols, leaving the burden of translating data in the appropriate format and transferring them to technicians familiar with those technological details (and who only need to provide links once per protocol). Third, users specify their own scenarios through rules that observe events produced in the environment and trigger reactions when relevant conditions are met. For that purpose, we rely on TRex \cite{cugola-12}, a powerful, decentralised \CEP engine and we automatically generate the necessary code transparently. Our solution takes advantages of Model-Driven Engineering (\MDE) to design a \DSL that is simple enough while capturing the relevant concepts appropriately and making it flexible enough to rebuild prototypes as the language evolves. In particular, our prototype currently relies upon a textual syntax, but we plan to design a more intuitive visual syntax for end users. 

Despite the promising results we experimented while using \IOTDSL on small examples with our industrial partners, we acknowledge that many challenges remain. First, reconciling high-level descriptions with low-level devices' \textsc{Api}s, as well as ensuring proper configuration of protocols from declarative intentions, necessitates \emph{glue code} that is not trivial. Fortunately, many initiative already exist, \textit{e.g.} we could rely on platforms like OpenRemote (\url{http://www.openremote.com}), SmartThings ({\url{https://www.smartthings.com}) or EnOcean (\url{https://www.enocean.com}) that abstract away several widely used protocols under the same \textsc{Api}. Here again, the use of dedicated \DSLS  could help designing robust and automatic solutions for these technical challenges. Second, our prototype is still at its early stage of development and many improvement directions remain. We must assess the relevance of our \DSL against various users and bigger cases to come up with a solution that can be widely adopted. Also, the aforementioned technical challenges need an appropriate answer to widen the automation capabilities of our solution, making it more relevant and usable. Deepening the understanding of each field (low-level devices \textsc{Api}s and communication protocols) would help implementing interesting, reusable solutions. Finally, at the long run, integrating specific properties of \IOT systems to guide the code generation while ensuring non-functional properties is necessary in such distributed and vulnerable \IOT systems.

\vfill
\bibliographystyle{apalike}
\bibliography{./Springer2017}


\end{document}