\section{Introduction}
\label{sec:Introduction}

Facing the explosion of available connected devices, many vendors are jumping into the market, proposing a large spectrum of products ranging from connected devices to associated end-user services~\cite{lee-15}. This results in a wide heterogeneity in software and hardware implementations, as well as an ever growing list of concerns and opportunities in terms of interoperability, data management, privacy and scalability~\cite{chaqfeh-12}.

As the Internet of Thigs (\IOT) infiltrates many aspects of people's life through their cars, heating systems, phones and so forth, a critical challenge is to provide end-users the possibility to benefit from the plethora of connected devices and configure them for their particular needs. From the recent research conducted at the University of Namur, we identified how difficult it is to make \textit{things} cooperate, and describe things configurations, because of the major influence of distinct technologies. In order to hide vendor-specific implementation details, we target a dedicated technology-agnostic environment to adapt and combine \IOT solutions.

Model-Driven Engineering (\MDE) has been recognised during the last decade as a software engineering technique dedicated to the design, management and evolution of computer languages enabling automatic generation of production code, diverse types of analysis and early verifications. Following this trend, we introduce \IOTDSL, a prototype Domain-Specific Language (\DSL) meant to capture \IOT devices capabilities and their deployment configurations, while providing a declarative way to end-users, letting them achieve their own scenarios.

\noindent
\textbf{Outline.} We start in Section \ref{sec:Context-Challenges} by presenting archetypal scenarios of \IOT devices usage to motivate why and how it becomes important to bring end-users back in control of the devices in their own domestic environment. We then extract the main crucial \IOT challenges specific to the use of \DSLS and \MDE techniques to realise this vision. In Section \ref{sec:IoTDSL}, we introduce \IOTDSL, our prototype \DSL to specify and interconnect devices in an intuitive and general way, and illustrate its use through a typical example. We overview in Section~\ref{sec:RW} the use of \DSLS for \IOT, comparing existing approaches with ours and assessing them against the challenges we identified; then conclude in Section \ref{sec:Conclusion} and present the main lines of work ahead to transform our prototype in a fully functional \DSL.
