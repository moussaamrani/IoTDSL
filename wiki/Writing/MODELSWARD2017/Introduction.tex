\section{Introduction}
\label{sec:Introduction}

The amount of connected devices around the world in the so called \textit{Internet of Things} is rapidly growing~\cite{gartner-15}. This \IOT market is clearly exploding and many vendors are jumping into with a wide range of products going from connected devices to associated end-user services~\cite{lee-15}. From that multitude of actors, an heterogeneity in software and hardware implementations has been raised, as well as a list of major concerns and opportunities in terms of interoperability, data management, privacy and scalability, for example~\cite{chaqfeh-12}.

As the \IOT infiltrates many aspects of people's life through their cars, heating systems, phones, and so forth, a critical challenge is to provide end-users the possibility to benefits from the plethora of connected devices and configure them for their particular needs. From past research conducted at the University of Namur, we identified how difficult it is to make \textit{things} cooperate, and describe configurations because of the major influence of distinct technologies. In order to hide vendor-specific implementation details, we target a dedicated technology-agnostic environment to adapt and combine \IOT solutions.

Upon the Model Driven Engineering initiative that offers facilities to create dedicated abstraction languages, generate code from models, define and run many kinds of verifications, we introduce \IOTDSL, a three-part prototype language meant to specify \IOT architectures using types, configurations and behavioural specification expressed as business rules.

The present paper is first devoted to draw a summary of the many challenges for the conception of a \DSL for the \IOT, in Section~\ref{sec:Context-Challenges}. Second, in Section~\ref{sec:IoTDSL}, we introduce \IOTDSL, a prototype \DSL to specify and interconnect \textit{things}. Next, we cross over existing specific languages and we compare to each other regarding the identified challenges in Section~\ref{sec:RW}. We finally sum up our contribution in Section~\ref{sec:Conclusion} and present the work yet to be done to go from the present prototype language to a fully fledged \DSL.