\section{Introduction}
\label{sec:Introduction}

Facing the explosion of new connected devices available, many vendors are jumping in the market, proposing a large spectrum of products ranging from connected devices to associated end-user services~\cite{lee-15}. This results in a wide heterogeneity in software and hardware implementations, as well as a list of major concerns and opportunities in terms of interoperability, data management, privacy and scalability, for example~\cite{chaqfeh-12}.

As the \IOT infiltrates many aspects of people's life through their cars, heating systems, phones, and so forth, a critical challenge is to provide end-users the possibility to benefits from the plethora of connected devices and configure them for their particular needs. From past research conducted at the University of Namur, we identified how difficult it is to make \textit{things} cooperate, and describe configurations because of the major influence of distinct technologies. In order to hide vendor-specific implementation details, we target a dedicated technology-agnostic environment to adapt and combine \IOT solutions.

Model-Driven Engineering (\MDE) has been recognised during the last decade as a software practice and a set of technologies dedicated to the design, management and evolution of computer languages from which it becomes possible to automatically generate production code, but also perform various analysis and early verifications. Following this trend, we introduce \IOTDSL, a prototype Domain-Specific Language (\DSL) meant to capture \IOT devices capabilities and their deployment configurations, while offering end-users a declarative way to express how these things interact to realise their own specific scenarios.

This paper starts by identifying in Section~\ref{sec:Context-Challenges} the \IOT challenges specific to the use of \DSLS and \MDE techniques for providing end-users control over their devices. We introduce in Section~\ref{sec:IoTDSL} \IOTDSL, our prototype \DSL to specify and interconnect devices in an intuitive and general way. We then in Section~\ref{sec:RW} overview the use of \DSLS for \IOT, comparing existing approaches with ours and assessing them against the challenges we identified. Section~\ref{sec:Conclusion} concludes and presents the main lines of work ahead to transform our prototype in a fully functional \DSL.