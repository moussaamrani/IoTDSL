\section{Context \& Challenges}
\label{sec:Context-Challenges}

\IOT devices could be used in many different situations for many different purposes. Consider these two examplars for smart homes:
\begin{enumerate}
	\item Alice leaves in a house equipped with many devices: a door and window lock detectors at key locations in the house, many devices that control the house's light, temperature and humidity, as well as security devices for detecting smoke and carbon dioxide, and many entertainment devices for TV, music, and sport. She is an active, working person, so she reconfigures her home devices often to accommodate her changing lifestyle: for example during winter, she exercices at home and needs her bathroom and living room warm enough, but not too much; while during summer, she likes walking and running outside, and would like her home to stay safe and to be notified of any intrusion. 
	
	\item Bob is 58 years old and leaves alone at home, while his children have a family life on the other side of the city. Like many elderly, Bob suffers from ageing diseases, but prefers to stay in a familiar environment rather than leaving in an institution. His house is equipped with devices that detects potential falls that can be dramatic for him, and with traditional entertainment devices such as a smart TV and phone. He wants to feel safe at home: if he falls and cannot stand up anymore in the bathroom, he would like to warn his family and notify daycare nurses to help him quickly cope with these situations. 
\end{enumerate}
These typical situations share two commonalities. First, both houses integrate common connected devices fulfilling the same functionalities (e.g., monitoring the temperature, or sending messages after a situation recognised as abnormal is detected), but are likely different in terms of vendors, communication protocols and detailed properties (for example, a temperature sensor could be paired with a heating system to maintain a given temperature inside the house; while in other configurations, both devices are clearly separated). Second, the way these devices interact highly depends on the end user, and more likely for the same user, on the present context: Alice's exemplar explicitly calls for different actions for different periods during the year; while Bob's examplar can easily be replicated in another house, but with different sets of concerns. 

We argue that a good way of capturing those potential variations of devices and situations is to provide to end users (be they home inhabitants like Alice, or technicians in charge of equipping house like Bob's) a \DSL that possess the following components:
\begin{description}
	\item[Device Description] A precise inventory of the devices used in a specific deployment as well as the high-level capabilities of these devices, described in terms that are immediately understandable by end-users, as opposed to conveying technical details about how those devices precisely execute;
	
	\item[Network Description] A way to capture where each device is located and how it is possible to communicate with it, in order to receive or send data to it;

	\item[Dynamics] A way to describe the interactions wished by end-users, i.e. how to leverage the functionalities of the devices to effectively realise one or several scenarios that are convenient for the end-users.  
	
\end{description}
Those components are obviously not sufficient for obtaining a fully-fledged solution that becomes adaptable to any situation, but they still represent the necessary steps for providing end-users the capacity to manipulate a collection of devices directly without resorting on specific technologies.

Nevertheless, the definition of such a \DSL raises many questions directly related to the many hypotheses assumed by those \DSL components.


















\begin{description}
	\item[Capability Discovery] Providing the ability to drive interconnected devices assumes the capacity of automatically discovering devices' capabilities in a standardised and uniform way. Similar processes happened for other technologies: for example, an \textsc{Usb} device plugged into a computer automatically exposes its nature (e.g., a pointing or video device) and capabilities. Classifying such capabilities could be useful to build an ontology of normalised capabilities that could result in powerful software \textsc{Api}s for manipulating devices. 
		
	\item[Complex Event Processing (\textsc{Cep})] Letting end-users manipulate devices through their low-level capability interfaces could lead to confusion and accidental complexity for defining usage scenarios. Rather, providing a way of reifying low-level internal computations inside devices into high-level events could help end-users leverage the complexity of devices networks and pave the way to freely and transparently manipulate them. Since \textsc{Cep} consists of deriving meaningful conclusions from a stream of events occurring within a system and of responding to them as quickly as possible, it provides a solution for extracting meaningful events from low-level computations. However, for a solution to be complete and useful, the reverse direction should be addressed: high-level actions should be adequately translated into low-level devices' actuations. 
		
	\item[Protocol Interoperability] A domotic solution with heterogeneous devices would often integrate devices from various constructors, thus communicating through multiple communication protocols. In order to make them communicate efficiently, without forcing end-users to stick with one constructor that can dictate costs and restrictions without any control, a powerful \textsc{Dsl} should provide ways for interoperability over multiple communication protocols, without forcing end-users to understand the protocols' intricacies, version evolution, and restrictions.
	
	\item[Scalability]
	
	\item[Data Management]  
	
	\item[Non-Functional Properties] A powerful \textsc{Dsl} should encompass typical non-functional properties of device networks to ensure long-life and secure realisation of scenarios. \emph{Performance} is crucial, and depends both on the devices capabilities but also on the quality of the network communications: any source of latency could have a dramatic impact that can lead to critical situations. \emph{Resource availability}, both in terms of computation and memory capability, but also in terms of energy is another crucial bottleneck for the adoption of \textsc{Dsl}s as a solution for defining scenarios: the code generated from the \textsc{Dsl} should not overload the devices with repetitive communications or unnecessary computations that would drain the device battery. \emph{Security} is yet another concern with respect to two aspects. First, sensitive data could be exposed through the communication network, endangering users privacy. Second, functionalities could be locked to be used only by authorised users.

\end{description}

