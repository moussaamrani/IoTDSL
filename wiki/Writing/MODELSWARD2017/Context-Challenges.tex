\section{Context \& Challenges}
\label{sec:Context-Challenges}

\IOT devices could be used in many different situations for many purposes. Consider these two examplars for smart homes:
\begin{enumerate}
	\item Alice leaves in a house equipped with a number of devices: door and window lock detectors spread at key locations in the house, devices that control the house's light, temperature and humidity, security devices for detecting smoke and carbon dioxide, as well as a plethora of entertainment devices for TV, music, and sport. She is an active, workwoman, so she often reconfigures her home devices to accommodate her changing lifestyle: for example during winter, she's staying inside, mainly teleworking, and needs her bathroom and living room to stay warm enough, but not too much; while during summer, she spends more time outside and she would like her home to stay safe, being notified of any intrusion.
	
	\item Bob is 68 years old and leaves alone at home, since his children have their own family lives on the other side of the city. As many elderly, Bob suffers from ageing diseases, but prefers to stay in a familiar environment rather than leaving in an dedicated institution. His house is equipped with devices monitoring potential falls that can have tragic consequences for him. It is also equipped with nowadays entertainment devices such as a smart TV and phone. He wants to feel safe at home: if he falls and cannot stand up anymore in the bathroom, he would like to warn his family and notify daycare nurses to help him quickly cope with these situations.
\end{enumerate}

These typical configurations share two commonalities. First, both houses integrate common connected devices fulfilling the same functionalities (\textit{e.g.}, monitoring the temperature, or sending messages after the detection of an abnormal situation, but are likely different in terms of vendors, communication protocols and detailed properties. For example, a temperature sensor could be paired with a heating system to maintain a given temperature inside the house, while in other configurations, both devices are clearly separated. Second, the way these devices interact with each others highly depends on the end user, and more likely for the same user, on the actual context. On the one hand, in Alice's situation, different actions on the same devices are required throughout the year. On the other hand, Bob's situation can be easily replicated to another house, but with different sets of concerns related to ageing diseases.

We argue that a good way of capturing those potential variations of devices and situations is to provide to end users, \textit{i.e.} home inhabitants like Alice, or technicians in charge of equipping houses like Bob's one, a \DSL that provides the following features:

\begin{description}
	\item[Device Description] A precise inventory of the devices used in a specific deployment as well as the high-level capabilities of these devices, described in terms that are immediately understandable by end-users, as opposed to conveying technical details about how those devices precisely execute;
	
	\item[Network Description] A way to capture where each device is located and how it is possible to communicate with it, in order to receive or send data to it;

	\item[Dynamics] A way to describe the interactions wished by end-users, \textit{i.e.} how to leverage the functionalities of the devices to effectively realise one or several scenarios that are convenient for the end-users.  
	
\end{description}
Those features are obviously not sufficient to obtain a fully-fledged solution that becomes adaptable to any situation, but they still represent necessary steps for providing end-users the capacity to manipulate a collection of devices directly without resorting on specific technologies.

Nevertheless, the definition of such a \DSL raises numerous questions directly related to the many hypotheses assumed by those \DSL features.

\begin{description}
	\item[Capability Discovery] Providing the ability to drive interconnected devices assumes the capacity of automatically discovering devices' capabilities in a standardised and uniform way. Similar processes exist for other technologies. For example, an \textsc{Usb} device plugged into a computer automatically exposes its nature (\textit{e.g.}, a pointing or video device) and capabilities. Classifying such capabilities could be useful to build an ontology of normalised functions that could result in powerful software \textsc{Api}s to manipulate devices. 
		
	\item[Complex Event Processing (\textsc{Cep})] Letting end-users deal with devices through their low-level capability interfaces could lead to confusion and accidental complexity for defining usage scenarios. Rather, providing a way of reifying low-level device computations into high-level events could help end-users leverage the complexity of devices networks and pave the way to manipulate them freely and transparently. Since \textsc{Cep} consists of deriving meaningful conclusions from a stream of events occurring within a system and responding to them as quickly as possible, it provides a solution to extract meaningful events from low-level computations. However, for a solution to be complete and useful, the reverse operation should be addressed: high-level actions should be adequately translated into low-level devices' actuations. 
		
	\item[Protocol Interoperability] A domotic solution with heterogeneous devices would often integrate devices from various providers, thus communicating through multiple communication protocols. In order to make them communicate efficiently, without forcing end-users to stick with one constructor that can dictate costs and restrictions without any control, a powerful \DSL should provide ways for interoperability over multiple communication protocols, without requiring end-users to understand the protocols' intricacies, versions and technical restrictions.
	
	\item[Scalability] As the number of devices and providers increases, the amount of connected devices is expected to rise exponentially. When updating existing IoT configurations, current solutions may not collapse when adding more elements to them. Furthermore, a \DSL must provide a way to absorb scalability problems, hiding as much as possible purely technical constraints regarding increases in size and complexity of operating configurations. 
	
	\item[Data Management] Analogously to scalability issues, the massive increase in connected devices will produce more and more data to be processed, stored and, for some of them, post processed~\cite{ma-13,lee-15}. More data means seemingly more storage capabilities and the required space to handle such a flow of information will be at its highest ever. Furthermore, the multiplication of available (sensors) sources is creating a whole new world of data processing and mining possibilities, but also a profusion of divergent concrete data types that sooner or later must be mapped to equivalent concepts.
	
	\item[Non-Functional Properties] A powerful \DSL should encompass typical non-functional properties of device networks to ensure long-life and secure realisation of scenarios. \emph{Performance} is crucial, and depends both on the devices capabilities but also on the quality of the communication network: any source of latency could have dramatic impacts that can lead to critical situations. \emph{Resource availability}, both in terms of computation and memory capability, but also in terms of energy is another crucial bottleneck for the adoption of \textsc{Dsl}s as a solution for defining scenarios. The code generated from the \DSL should not overload the devices with repetitive communications or unnecessary computations that would drain the device's battery. \emph{Security} is yet another concern with respect to two aspects. First, sensitive data could be exposed through the communication network, endangering users privacy. Second, some functionalities could be locked and only accessible to authorised users.

\end{description}

