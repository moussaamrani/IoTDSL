\section{Conclusion and Future Work}
\label{sec:Conclusion}

In this paper, based on recent research, we summarised the needed features as well as major challenges for the adoption of a specific language for \IOT systems. We introduced \IOTDSL, a domain specific language, designed to tackle the various challenges identified as crucial features for the adoption of \IOT technologies and \textsc{Dsl}s, \textit{e.g.} in terms of capability discovery, event processing or interoperability. In \IOTDSL, devices are described in a common \textit{Classifier} - \textit{Type} - \textit{Instance} layered specification that allows modellers to easily distinguish between typing and running instances concepts. A second benefit of that separation is the ability to create reusable \textit{libraries} of conceptual components, so \IOT devices by extension. By raising the abstraction level and hiding technical constraints to end-users, \IOTDSL is designed to serve as a handy modelling mean for them. Moreover, by abstracting the communication complexity into dedicated modelling construct, it is aimed to facilitate future evolution and re-configuration of existing solutions, since device descriptions may be retrieved from a centralized gateway.

Behavioural specifications are key aspects in \IOTDSL. We decided to rely on event-based \textit{rules} instead of state machines for their conciseness and readability, though rules may be converted to state machines automatically. Together with high-level \textit{operations} descriptions, they offer a more concise mean to describe dynamic aspects of \IOT configurations, without loosing formal model verification capabilities.
