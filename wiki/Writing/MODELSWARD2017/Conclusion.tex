\section{Conclusion and Future Work}
\label{sec:Conclusion}

In this paper, we presented a prototype \DSL named \IOTDSL, designed for capturing the definition of devices' capabilities and their concrete deployment in specific installations. It provides a declarative language based on rules to enable end-users to define their own scenarios for driving their installation, by manipulating devices through high-level concepts. We also identified key challenges from the litterature specific to \DSL engineering in the area of \IOT. 

At the heart of \IOTDSL is a clear separation of the three main concerns each \DSL for \IOT need to address. By raising the abstraction level and offering a conceptual view of devices' capabilities, \IOTDSL promotes reuse through dedicated device libraries, strongly suggesting a standardisation of interfaces like the ones already existing in other domains (for example, the many computer devices using \textsc{Usb}). Since \IOTDSL is developed with \MDE tools, switching to a visual interface is easier than when a \DSL relies on General-Purpose Programming Languages. The communication between devices is a volatile domain, with new protocols emerging every year. By only declaring how things communicate, we push the burden of translating / extracting data from low-level protocols to high-level interfaces towards the technicians in charge of defining and understanding such protocols. However, this task is done only once per protocol, and can reuse the experience and techniques already existent in other areas. For the user, this aspect enforces live reconfiguration of networks of things, as we already experience in our daily life. 

Despite the promising results we experience while using our \DSL on small example with our industrial partners, we aknowledge that many challenges remain. First, reconciling high-level device capabilities with low-level complex communication framework embedded for the plethora of devices available will require Complex Event Processing, a now mature field with powerful techniques. Second, evaluating our declarative sublanguages, for network configurations and business rules, on real-size deployments will provide us insight on how to improve each sublanguage and identify which patterns need to be integrated to facilitate such definitions. Finally, non-functional properties need to be enforced through appropriate code generation both in a centralised and distributed configurations.

 

%In this paper, based on recent research, we summarised the needed features as well as major challenges for the adoption of a specific language for \IOT systems. We introduced \IOTDSL, a domain specific language, designed to tackle those various challenges, \textit{e.g.} in terms of capability discovery, event processing or interoperability. In \IOTDSL, devices are described in a common \textit{Classifier} - \textit{Type} - \textit{Instance} layered specification that allows modellers to easily distinguish between typing and running instances concepts. A second benefit of that separation is the ability to create reusable \textit{libraries} of conceptual components, so \IOT devices by extension. By raising the abstraction level and hiding technical constraints to end-users, \IOTDSL is designed to serve as a handy modelling mean for them. Moreover, by abstracting the communication complexity into dedicated modelling construct, it is aimed to facilitate future evolution and re-configuration of existing solutions, since device descriptions may be retrieved from a centralized gateway.
%
%Behavioural specifications are key aspects in \IOTDSL. We decided to rely on event-based \textit{rules} instead of state machines for their conciseness and readability, though rules may be converted to state machines automatically. Together with high-level \textit{operations} descriptions, they offer a more concise mean to describe dynamic aspects of \IOT configurations, without loosing formal model verification capabilities.
%
%However, the prototype language is at early development stage and some features are currently missing or under development. First, the expression language must be validated on a real world examples and the mapping layer in charge of reconciling the abstract definitions and the technological APIs. Second, the complex composition of rules must be further investigated. Last, we plan to investigate on logic migration possibilities from the gateways to nodes with sufficient capacity. 