\section{Introduction}
\label{sec:Introduction}

Facing the explosion of available connected devices, many vendors are jumping into the market, proposing a large spectrum of products ranging from connected devices to associated end-user services~\cite{lee-15}. This results in a wide heterogeneity in software and hardware implementations, as well as an ever growing list of concerns and opportunities in terms of interoperability, data management, privacy and scalability~\cite{chaqfeh-12}.

As the Internet of Things (\IOT) infiltrates many aspects of people's life through their cars, heating systems, phones and so forth, a critical challenge is to provide end-users the possibility to benefit from the plethora of connected devices and configure them for their particular needs. Moreover, user-defined workflows usually expressed at a high level of abstraction must be somehow translated into runnable entities and the orchestration between many devices usually interconnected into a single workflow is a non trivial task. In order to hide vendor-specific implementation details, we target a dedicated technology-agnostic environment to adapt and combine \IOT solutions whose external behaviours have been expressed in a specific language.

Model-Driven Engineering (\MDE) has been recognised during the last decade as a software engineering technique dedicated to the design, management and evolution of computer languages enabling automatic generation of production code, diverse types of analysis and early verifications. Following this trend, we introduce \IOTDSL, a prototype Domain-Specific Language (\DSL) meant to capture \IOT devices capabilities and their deployment configurations, while providing a declarative way to end-users, letting them achieve their own scenarios. 

Event-based systems have appeared in many domains and \IOT infrastructures are well known examples of such systems~\cite{muhl-06,cristea-11}. Rule-based systems are widely used in a vast range of domains like finance~\cite{schultz-09}, disaster monitoring~\cite{broda-09}, social threats discovery~\cite{baran-13} and so forth. Rules are particularly suitable to express composition of events because of their declarative nature and their high-level of abstraction, thus in \IOTDSL, user scenarios are expressed in a rule-based language that empowers reusability and automatic translation into a runnable Complex Event Processing (\CEP)-based language~\cite{cugola-12}.

\noindent
\textbf{Outline.} We start in Section \ref{sec:Motivation} by presenting an archetypal scenario of a smart house to highlight the usefulness to bring end-users back in control of their own domestic \IOT environment. We also extract the crucial \IOT challenges specific to the use of \DSLS and \MDE techniques to realise this vision. In Section \ref{sec:IoTDSL}, we introduce \IOTDSL, our prototype \DSL to specify and interconnect devices in an intuitive and general way and illustrate its benefits through use cases extracted from our smart house example. Then, in Section~\ref{sec:CG}, we detail how we translate \IOTDSL rules into a concrete \CEP engine and how we generate simulation facilities meant to test and validate the \IOT deployment. We discuss our approach and the remaining challenges to tackle in Section~\ref{sec:ChallengeDiscussion}. We overview in Section~\ref{sec:RW} the use of \DSLS for \IOT, comparing existing approaches with ours and assessing them against the challenges we identified. Finally, we conclude in Section \ref{sec:Conclusion} and present the main lines of work ahead to transform our prototype in a fully functional \DSL framework.
