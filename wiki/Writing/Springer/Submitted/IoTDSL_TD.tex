\subsection{Type Definition}
\label{sec:IoTDSL-TD}

The first task is to provide a description of which capabilities each device possess, how each device may provide information about the environment through a \emph{sensing} operation, and how it could react and influence it through \emph{actuations}. Our framework currently requires that an advanced user extract  relevant information regarding devices' exposed features, but it is flexible enough to accommodate automation in the future, so that such pieces of information could be automatically extracted from pre-existing devices databases (either from a knowledge database the \IOT system is connected to, or from a library of \emph{off-the-shelf devices}).

The concepts dedicated to type definition are shown in Figure~\ref{fig:IoTDevice-MM} (top-right part in green background). This part is similar to the notion of \textsf{Classifier} in \textsc{Mof}-like languages: a \textsf{Type} is either a \textsf{PrimitiveType}, or a user-defined \textsf{DeclaredType}. We distinguish between general \textsf{Gateway}s, which centralise information and processing, and \textsf{Node}s deployed in the environment and communicating with \textsf{Gateway}s, and which possess capabilities to interact with the environment. A \textsf{Capability} is basically a parametrised event that drives the node to either capture data from the environment, act on it, or perform both. This abstract view of a ``thing'' allows us to manipulate any device at a high level of abstraction, exhibiting a clean and uniform interface for end-users based on device capabilities. Since \textsf{Node}s are \textsf{Type}s themselves, they may be referenced as parameters for the purpose of dynamic discovery across devices.

Listing~\ref{lis:RE-TypeDeclarations} illustrates how the devices in Figure~\ref{fig:scenario} are declared in \IOTDSL. Each device is introduced by the keyword \textsf{device}, possesses a name and lists capabilities that correspond to reporting events (\textsf{sensing}) or operating over the environment (\textsf{actuating}). 

\begin{table}
	\begin{minipage}[b]{.45\textwidth }%
		\begin{lstlisting}[language=iotdsl]	
gateway Middleware
device DoorDetector {
	sensing opened()
	sensing closed()
}
device MotionDetector {
	sensing moving()
}
device ToggleSwitch {
	sensing toggled()
}
		\end{lstlisting}
	\end{minipage}\hfill%
	\begin{minipage}[b]{.45\textwidth}
		\begin{lstlisting}[language=iotdsl, firstnumber=12]
device LightSensor {
	sensing light()
}
device LightBulb {
	actuating on()
	actuating off()
}	
device Alarm {
	actuating sound()
}
		\end{lstlisting}
		\vspace{.3cm}
	\end{minipage}
	\caption{Type declarations in \IOTDSL: capabilities as high-level events.}
	\label{lis:RE-TypeDeclarations}
\end{table}

Any IoT system should declare a special device, introduced with the keyword \textsf{gateway}, that centralises data from all devices connected to it, as we will show in Section \ref{sec:IoTDSL-NetworkConfiguration}. This device will be responsible of the event orchestration and will host the \CEP engine that embeds the implementation of the business rules. Also note that the above model is the \textit{user-defined} part of \IOTDSL. In the background, abstract events attached to all devices will need to be mapped to concrete low-level \textsc{Api}s events using a dedicated mapping language that is out of the scope of this paper.

