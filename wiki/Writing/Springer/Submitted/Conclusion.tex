\section{Conclusion and Future Work}
\label{sec:Conclusion}

In this paper, we identified key challenges from the literature that are specific to \DSL engineering in the area of \IOT. On top of these challenges and features, we presented a prototype \DSL named \IOTDSL, designed for capturing the definition of devices' capabilities and their concrete deployment in specific configurations. It provides a declarative language based on business rules where end-users may define their own scenarios by manipulating devices through high-level concepts. Those rules are then translated and injected into a Complex Event Processing (\CEP) engine responsible to evaluate the occurrences of events and firing appropriate reactions. Together with the TRex-compliant rules, a set of Java source files are generated to simulate the whole network under consideration.  

At the heart of \IOTDSL is a clear separation of three main concerns that any \DSL for \IOT should address. By raising the abstraction level and offering a conceptual view of devices' capabilities, \IOTDSL promotes reuse through dedicated device libraries, strongly suggesting a standardisation of interfaces like the ones already existing in other domains (for example, the many computer devices using \textsc{Usb}). Since \IOTDSL is developed with \MDE tools, adding a visual syntax to structural definition of devices and networks is almost straightforward. The communication between devices is a volatile domain, with new protocols emerging every year. By only declaring how things communicate, we push the burden of translating / extracting data from low-level protocols to high-level interfaces towards technicians in charge of defining and understanding such protocols. However, this task is done only once per protocol, and can reuse the experience and techniques already available in other areas. For users, this aspect enforces live reconfiguration of networks of things, as we already experience in our daily life. 

Despite promising results we experienced while using our \DSL on small examples with our industrial partners, we acknowledge that many challenges remain. First, reconciling high-level device capabilities with low-level complex communication frameworks available for the plethora of devices will to write \textit{<< glue code >>} for data type translations and concrete bindings of devices to our generated middleware agent. Second, evaluating our declarative sub-languages, for network configurations and business rules, on large-scale deployments will provide us insight on how to improve each sublanguage and identify which patterns need to be integrated into libraries to facilitate such definitions. Finally, non-functional properties need to be enforced through appropriate code generation both in a centralised and distributed configurations.
