\section{Conclusion and Future Work}
\label{sec:Conclusion}

The Internet of Things promotes the usage of various interconnected devices that pro\-mise to help end users achieve more automation in daily life recognisable scenarios. As such, a smart home could automatically close doors and switch off lights when the inhabitants leave, or facilitate life routines by assisting in tasks like preparing coffee just in time. The flip side of the coin resides in the ever growing spectrum of connected devices proposed by vendors that spot the market as a good opportunity to make profit, without ensuring a minimal interoperability between their products and those available from other merchants. As a result, the promise seems far from happening in the near future without powerful solutions to catalogue connected devices, to make them exchange relevant data and act in a disciplined way. Furthermore, without bringing end users at the heart of their own story and providing them tools to define, drive and adapt their own scenarios, vendors will always keep a grasp at the \IOT market.

In this paper, we explore a first step towards achieving this large challenge by proposing a prototype that aims at raising end users as main actors of how smart home devices interact for their own needs. Aware of the many challenges surrounding \IOT systems including reusability, interoperability, scalability and non-functional properties, we designed our solution as an evolving and decentralised tool that allows end users to specify their own scenarios based on so-called rule-based definitions. Our prototype takes the form of a Domain-Specific Language (\DSL) associated with a code generator that produces executable code designed to run on a Complex Event Processing (\CEP) engine, as well as emulation code dedicated to simulate the whole system before effectively deploying it with concrete devices.

Our prototype \IOTDSL clearly separates three necessary aspects when describing solutions for \IOT systems. First, it captures devices capabilities as high-level events that are meaningful to end users, thus hiding the intricacies of low-level manipulation of \DSLS into our platform. Second, it describes device interconnections in a declarative language with predefined communication protocols, leaving the burden of translating data in the appropriate format and transferring them to technicians familiar with those technological details (and who only need to provide links once per protocol). Third, users specify their own scenarios through rules that observe events produced in the environment and trigger reactions when relevant conditions are met. For that purpose, we rely on TRex \cite{cugola-12}, a powerful, decentralised \CEP engine and we automatically generate the necessary code transparently. Our solution takes advantages of Model-Driven Engineering (\MDE) to design a \DSL that is simple enough while capturing the relevant concepts appropriately and making it flexible enough to rebuild prototypes as the language evolves. In particular, our prototype currently relies upon a textual syntax, but we plan to design a more intuitive visual syntax for end users. 

Despite the promising results we experimented while using \IOTDSL on small examples with our industrial partners, we acknowledge that many challenges remain. First, reconciling high-level descriptions with low-level devices' \textsc{Api}s, as well as ensuring proper configuration of protocols from declarative intentions, necessitates \emph{glue code} that is not trivial. Fortunately, many initiative already exist, \textit{e.g.} we could rely on platforms like OpenRemote (\url{http://www.openremote.com}), SmartThings ({\url{https://www.smartthings.com}) or EnOcean (\url{https://www.enocean.com}) that abstract away several widely used protocols under the same \textsc{Api}. Here again, the use of dedicated \DSLS  could help designing robust and automatic solutions for these technical challenges. Second, our prototype is still at its early stage of development and many improvement directions remain. We must assess the relevance of our \DSL against various users and bigger cases to come up with a solution that can be widely adopted. Also, the aforementioned technical challenges need an appropriate answer to widen the automation capabilities of our solution, making it more relevant and usable. Deepening the understanding of each field (low-level devices \textsc{Api}s and communication protocols) would help implementing interesting, reusable solutions. Finally, at the long run, integrating specific properties of \IOT systems to guide the code generation while ensuring non-functional properties is necessary in such distributed and vulnerable \IOT systems.
