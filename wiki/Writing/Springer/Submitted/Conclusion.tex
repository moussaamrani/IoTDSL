\section{Conclusion and Future Work}
\label{sec:Conclusion}

The Internet of Things promotes the usage of various interconnected devices that promise to help end users achieve more automation in daily life recognisable scenarios: as such, a smart home could automatically close doors and switch off lights when the inhabitants leave; or facilitate life routines by assisting in tasks like preparing coffee just in time. The flip side of the coin resides in the ever growing spectrum of connected devices proposed by vendors that cease the market as a good opportunity to make profit, without ensuring a minimal interoperability between their products and those available from outside. As a result, the promise seems far from happening in the near future without powerful solutions to catalog connected devices, to make them exchange relevant data and act in a disciplined way. And without bringing the end users at the center of their own story and providing them the tools to define, drive and adapt their own scenarios, tool vendors will always keep a grasp at the \IOT market.

In this paper, we explore a first step towards achieving this large challenge, by proposing a prototype that aims at bringing the end users the main actor of how smart home devices interact for their own needs. Aware of the many challenges surrounding \IOT systems, including reusability, interoperability, scalability and non-functional properties handling like security or resource availability, we designed our solution as an evolutive, decentralised tool that allows end users specify their own scenarios based on so-called rule-based definitions. Our prototype takes the form of a Domain-Specific Language (\DSL) associated with a code generator that produces executable code designed to run on a Complex Event Processing (\CEP) engine, as well as emulation code that allows simulating the whole system before effectively deploying it with actual devices. 

Our prototype \IOTDSL clearly separate three aspects necessary to describe any solution for \IOT systems: it captures devices capabilities as high-level events meaningful for end users, thus hiding the intricacies of the low-level manipulation of \DSLS into our platform; it describes device interconnections through a declarative language specifying which protocols are used to communicate, leaving the burden of translating data in the appropriate format and transporting them to technicians aware of those technicalities (and who only need to provide links once per protocol); and users specify their own scenarios through rules that observe the environment, trigger when relevant events are detected, and react accordingly. We rely on TRex \cite{cugola-12}, a powerful, decentralised \CEP engine to manage event processing, and we automatically generate the necessary code transparently. Our solution relies on recent advances in Model-Driven Engineering (\MDE): designing a \DSL that is simple enough while capturing the relevant concepts appropriately is notoriously difficult for an emerging domain like \IOT, but \MDE provides techniques to support rapid evolution of language definitions and tool support to prototype solutions quickly. In particular, our prototype currently relies on a textual syntax, but we plan to design a visual syntax, more intuitive for end users, without extra burden since our language is already completely modelled. 

Despite the promising results we experimented while using \IOTDSL on small examples with our industrial partners, we we acknowledge that many challenges remain. First, reconciling high-level descriptions with low-level \textsc{Api}s devices manipulate, as well as ensuring proper configuration of protocols from declarative intentions, necessitates \emph{glue code} that is not trivial. Fortunately, many initiative already exist, e.g. we could rely on platforms like OpenRemote (\url{http://www.openremote.com}), SmartThings ({\url{https://www.smartthings.com}) or EnOcean (\url{https://www.enocean.com}) that abstract away several widely used protocols under the same \textsc{Api}. Here again, the use of dedicated \DSLS
 could help designing robust, automatic solutions for these technical challenges.

Our prototype is still at its early stage and many improvement directions remain. First, assessing the relevance of our \DSL against various users and bigger cases is essential for coming up with a widely adopted solution. Second, the technical challenges mentioned above need an appropriate answer to widen the automation capabilities of our solution, making it more relevant and usable. Deepening the understanding of each field (low-level devices \textsc{Api}s and communication protocols) would help implementing interesting, reusable solutions. Finally, at the long run, integrating specific properties of \IOT systems to guide the code generation while ensuring non-functional properties is necessary in such distributed and vulnerable \IOT systems.


%In this paper, we identified key challenges from the literature that are specific to \DSL engineering in the area of \IOT. On top of these challenges and features, we presented a prototype \DSL named \IOTDSL, designed for capturing the definition of devices' capabilities and their concrete deployment in specific configurations. It provides a declarative language based on business rules where end-users may define their own scenarios by manipulating devices through high-level concepts. Those rules are then translated and injected into a Complex Event Processing (\CEP) engine responsible to evaluate the occurrences of events and firing appropriate reactions. Together with the TRex-compliant rules, a set of Java source files are generated to simulate the whole network under consideration.  
%
%At the heart of \IOTDSL is a clear separation of three main concerns that any \DSL for \IOT should address. By raising the abstraction level and offering a conceptual view of devices' capabilities, \IOTDSL promotes reuse through dedicated device libraries, strongly suggesting a standardisation of interfaces like the ones already existing in other domains (for example, the many computer devices using \textsc{Usb}). Since \IOTDSL is developed with \MDE tools, adding a visual syntax to structural definition of devices and networks is almost straightforward. The communication between devices is a volatile domain, with new protocols emerging every year. By only declaring how things communicate, we push the burden of translating / extracting data from low-level protocols to high-level interfaces towards technicians in charge of defining and understanding such protocols. However, this task is done only once per protocol, and can reuse the experience and techniques already available in other areas. For users, this aspect enforces live reconfiguration of networks of things, as we already experience in our daily life. 
%
%Despite promising results we experienced while using our \DSL on small examples with our industrial partners, we acknowledge that many challenges remain. First, reconciling high-level device capabilities with low-level complex communication frameworks available for the plethora of devices will to write \textit{<< glue code >>} for data type translations and concrete bindings of devices to our generated middleware agent. Second, evaluating our declarative sub-languages, for network configurations and business rules, on large-scale deployments will provide us insight on how to improve each sublanguage and identify which patterns need to be integrated into libraries to facilitate such definitions. Finally, non-functional properties need to be enforced through appropriate code generation both in a centralised and distributed configurations.
