\section{Discussion \& Remaining Challenges}
\label{sec:Discussion}

The \IOTDSL framework has been designed to empower non-experts with facilities regarding the requirements of \textit{smart home} \IOT solutions. Coupled to \IOT devices and network specifications, the framework offers a rule-based language compilable into a concrete \CEP infrastructure in charge of the event orchestration. \IOTDSL users are then able to describe their own configurations and needs in terms of conditional events and reactions to the manifestation of such conditions.

Tracing back to the features and challenges identified in Section~\ref{sec:Motivation-Challenges}, \IOTDSL currently covers most of these aspects. We created a description language that captures devices capabilities at a high-level of abstraction, describing them as entities that produce and consume (typed) events, relieving end users from acquiring the low-level knowledge to manipulate them. Based on these specifications, users become able to represent smart home configurations easily by describing links between devices and declaring which protocols are used for communication from a set of predefined, widely used protocols. Device interactions are simply expressed as conditions triggering other events that would react on the environment, inducing physical actions on the real world. These three sublanguages defined in \IOTDSL cover the identified language components identified as necessary for any \DSL dedicated to \IOT systems.

%We created a layered description language where type of devices can be described at a high level of abstraction, without requiring knowledge in a particular technology. Objects are simply described in terms of producing and consuming (typed) events. From these specifications, modellers are able to represent \textit{smart home} configurations by linking devices to each other with a predefined set of abstract communication protocols. Interactions between multiple devices are then simply  expressed as conditions triggering other events. Those three sub-languages of \IOTDSL just cover the three identified needed features of a \DSL for \IOT.

Based on the literature, we also identified in Section \ref{sec:Motivation-Challenges} seven important challenges an \IOT modelling solution should tackle. At the current development stage, \IOTDSL fully addresses the following challenges. \IOTDSL sees devices through their high-level capabilities, which basically corresponds to an ontological device description: it describes the \emph{interface} of devices in a general way, thus facilitating \emph{Capability Discovery} as it becomes available. By separating device descriptions from how they are connected to each others, \IOTDSL empowers \emph{Reusability} of devices through different \IOT systems; furthermore, since partial configurations could be easily defined and imported in \IOTDSL, partial definitions and behavioural specifications could also be shared between various instalments. 

We notably rely on a powerful \CEP engine to handle event orchestrations and deliver physical actions through the system. Our framework processes abstract business rules and transforms them automatically into executable code, and produces sample code for simulation purposes. The \emph{scalability} issue is almost exclusively concentrated on the device intercommunication and the rule processing, making scalability issues rely almost completely on the \CEP engine. TRex, the engine we have chosen, already offers parallelisation mechanisms that we will leverage to divide monolithic solutions into smaller entities that would collaborate, assuming an \IOT system becomes too large to handle, so that part of the business rules could be deployed on distinct parts of the system to minimise the middleware workload. In turn, these could generate even more events through the network and result in saturating it with data transmissions: an appropriate tradeoff needs to be found between having more one-to-one communications, or grouping more logics inside one node.

There is however three other challenges that we scarcely target with \IOTDSL, even if they are partially encompassed in some way, mostly because they represent orthogonal aspects or concerns that comes after the domain targetted by our language. \IOTDSL targets the manipulation of device \emph{interactions}, but already provides, through \CEP, a limited paradigm for \emph{Data Manipulation} that scales. However, as usual, a dedicated \DSL could be more relevant since the operations required for such operations are somehow different. An interesting discussion would then to identify the interface needed to exchange information between data retrieved from device interactions and data processed offline with higher processing capabilities. For now, \IOTDSL hides the intricacies of \emph{protocol communication interoperability} by implementing simple connectors to each protocol we handle, and reusing existing infrastructures for protocols that we do not handle natively. However, it could be interesting to propose a generic infrastructure for protocol communication by separating the transport layer from the message representation. This is a specific expertise domain on its own that we will tackle later. For \IOTDSL to handle \emph{non-functional properties}, we first need to have precise description of the devices hardware properties, which is an active research domain on its own. From that, we could integrate information that would guide the automatic code generation process to specialise the code to either decentralise parts of the processing activities, or ensure better performance, or integrate best practices for security.

%\begin{description}
	%\item[Capability Discovery] \IOTDSL \textsl{capabilities} have been designed to enable dynamic discovery of devices \textit{interfaces} as a \textsl{capability} can handle devices as parameters.
	%
	%\item[Reusability] By separating devices' descriptions to network configurations, \IOTDSL empowers reusability of devices in different \IOT systems. Furthermore, as partial \IOTDSL models can be imported into other models, partial definition of networks as well as behavioural specifications can be reused throughout models.
	%
	%\item[Complex Event Processing] \CEP is the core of \IOTDSL by which the whole events orchestration is handled. Abstract business rules are automatically transformed into a runnable infrastructure and sample code is also generated by the framework for simulation purposes. We notably rely on a powerful \CEP engine that will allow us to even decentralise the middleware code into collaborating nodes in future development of the framework.	
	%
	%\item[Scalability] As we rely on \CEP for device communication, almost the whole scalability issue is on the \CEP engine's hand. But, as we have chosen our infrastructure carefully, current monolithic solution is dividable into smaller entities that will collaborate if an \IOT network becomes large, such that some (set of) business rules will be deployed on distinct entities, minimising the work load on the middleware, even needing no middleware at all and working in a fully-decentralised way.
	%
	%\item[Data Management] To cope with an increase of exchanged data, prioritisation and decentralisation policies can be handled by our framework. On the one hand, thanks to configuration possibilities of TRex, events retention policies may be configured. On the other hand, again because TRex supports distributed evaluation of events, the workload may be scattered on the whole network. However, replicating \CEP-capable nodes may produce more events and saturate the network with data transmissions, so an appropriate trade-off must be found between having more \textit{one-to-one} communications or grouping more logic inside one node.
	%
%\end{description}

%However, we scarcely focus on the remaining two challenges, even if they are partially encompassed in current language.

%\begin{description}
	%\item[Protocol Interoperability] Concrete connections of \textit{things} is not addressed at current stage of development. Mapping rules between abstract devices definitions and concrete specifications of objects are under investigation. Furthermore, as a fully \MDE approach, \IOTDSL should provide generation capabilities of partial \textit{glue} code to map between tangible objects and our \CEP middleware.
%
	%\item[Non-Functional Properties] As for communication-specific features, more general non-functional properties should be expressible in \IOTDSL, especially as we deal with high-level user requirements. This essential aspect is also under investigation to empower modellers with semantic refinements at the typing level and also at the technical mapping level we are currently considering.
%\end{description}