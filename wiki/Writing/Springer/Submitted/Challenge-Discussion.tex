\section{Discussion \& Remaining Challenges}
\label{sec:Discussion}

The \IOTDSL framework has been designed to empower non-experts with facilities regarding the requirements of \textit{smart home} \IOT solutions. Coupled to \IOT devices and network specifications, the framework offers a rule-based language compilable into a concrete \CEP infrastructure in charge of event orchestration. \IOTDSL users are then able to describe their own configurations and needs in terms of conditional events and reactions to the manifestation of such conditions.

Tracing back to the features and challenges identified in Section~\ref{sec:Motivation-Challenges}, \IOTDSL currently covers most of these aspects. We created a description language that captures devices capabilities at a high-level of abstraction, describing them as entities that produce and consume (typed) events, relieving end users from acquiring the low-level knowledge to manipulate them. Based on these specifications, users become able to represent smart home configurations easily by describing links between devices and declaring which protocols are used for communication from a set of predefined and widely used protocols. Device interactions are simply expressed as conditions triggering other events that may react on the environment, inducing physical actions on the real world. These three sublanguages defined in \IOTDSL cover the language components identified as necessary for any \DSL dedicated to \IOT systems.

Based on the literature, we also identified seven important challenges an \IOT modelling solution should tackle, most of them being considered in \IOTDSL. Devices are seen through their high-level capabilities, which basically correspond to an ontological device description: it describes the \emph{interface} of devices in a general way, facilitating \emph{Capability Discovery} as it becomes available. By separating device descriptions from how they are connected to each others, \IOTDSL empowers \emph{Reusability} of devices through different \IOT systems. Furthermore, since partial configurations may be defined and imported in \IOTDSL models, partial definitions and behavioural specifications may also be shared between various installation.

We notably rely on a powerful \CEP engine to handle event orchestrations and deliver physical actions through the system. Our framework processes abstract business rules and transforms them automatically into executable code and produces sample code for simulation purposes. The \emph{scalability} issue is almost exclusively concentrated in the device intercommunication and the rule processing, making the \CEP engine dealing with scalability issues. TRex, the engine we have chosen, already offers parallelisation mechanisms that we will leverage to divide monolithic solutions into smaller entities that would collaborate, assuming an \IOT system becomes too large to handle within a single engine instance, so that part of the business rules might be deployed on distinct parts of the system to minimise the middleware workload. In turn, these could generate even more events through the network and result in a congestion with data transmissions. An appropriate tradeoff needs to be found between having more \textit{one-to-one} communications or grouping more logics inside a particular node.

There is however three challenges that we scarcely target with \IOTDSL, even if they are partially encompassed in some way, mostly because they represent orthogonal aspects or concerns that comes after the domain targeted by our language. \IOTDSL targets the manipulation of device \emph{interactions}, but already provides, through \CEP, a limited paradigm for \emph{Data Manipulation} that scales. However, as usual, a dedicated \DSL could be more relevant since the operations required for such operations are somehow different. An interesting discussion would then to identify the interface needed to exchange information between data retrieved from device interactions and data processed offline with higher processing capabilities. For now, \IOTDSL hides the intricacies of \emph{protocol communication interoperability} by implementing simple connectors to each protocol we handle, and reusing existing infrastructures for protocols that we do not handle natively. However, it could be interesting to propose a generic infrastructure for protocol communication by separating the transport layer from the message representation. This is a specific expertise domain on its own that we will later tackle. For \IOTDSL to handle \emph{non-functional properties}, we first need to have precise description of the devices hardware properties, which is an active research domain on its own. From that, we could integrate information that would guide the automatic code generation process to specialise the code to either decentralise parts of the processing activities, ensure better performance or integrate best practices for security.