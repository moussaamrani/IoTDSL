\section{Discussion \& Remaining Challenges}
\label{sec:ChallengeDiscussion}

The \IOTDSL framework has been designed to empower non-experts with facilities regarding the requirements of \textit{smart home} \IOT solutions. Coupled to \IOT devices and network specifications, the framework offers a rule-based language compilable into a concrete \CEP infrastructure in charge of the event orchestration. \IOTDSL users are then able to describe their own configurations and needs in terms of conditional events and reactions to the manifestation of such conditions.

Tracing back to the features and challenges identified in Section~\ref{sec:Motivation}, \IOTDSL currently meet most of these aspects. We created a layered description language where type of devices can be described at a high level of abstraction, without requiring knowledge in a particular technology. Objects are simply described in terms of producing and consuming (typed) events. From these specifications, modellers are able to represent \textit{smart home} configurations by linking devices to each other with a predefined set of abstract communication protocols. Interactions between multiple devices are then simply  expressed as conditions triggering other events. Those three sub-languages of \IOTDSL just cover the three identified needed features of a \DSL for \IOT.

Based on the literature, we also identified seven important challenges a \IOT modelling solution should tackle. At current development stage, we completely address the following challenges.

\begin{description}
	\item[Capability Discovery] \IOTDSL \textsl{capabilities} have been designed to enable dynamic discovery of devices \textit{interfaces} as a \textsl{capability} can handle devices as parameters.
	
	\item[Reusability] By separating devices' descriptions to network configurations, \IOTDSL empowers reusability of devices in different \IOT systems. Furthermore, as partial \IOTDSL models can be imported into other models, partial definition of networks as well as behavioural specifications can be reused throughout models.
	
	\item[Complex Event Processing] \CEP is the core of \IOTDSL by which the whole events orchestration is handled. Abstract business rules are automatically transformed into a runnable infrastructure and sample code is also generated by the framework for simulation purposes. We notably rely on a powerful \CEP engine that will allow us to even decentralise the middleware code into collaborating nodes in future development of the framework.	
	
	\item[Scalability] As we rely on \CEP for device communication, almost the whole scalability issue is on the \CEP engine's hand. But, as we have chosen our infrastructure carefully, current monolithic solution is dividable into smaller entities that will collaborate if an \IOT network becomes large, such that some (set of) business rules will be deployed on distinct entities, minimising the work load on the middleware, even needing no middleware at all and working in a fully-decentralised way.
	
	\item[Data Management] 
	
\end{description}

However, we scarcely focus on the remaining two challenges, even if they are partially encompassed in current language.

\begin{description}
	\item[Protocol Interoperability]

	\item[Non-Functional Properties] 
\end{description}