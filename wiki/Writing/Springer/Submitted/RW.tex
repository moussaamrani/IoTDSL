\section{Related Work}
\label{sec:RW}

A series of overviews have been recently conducted on several aspects of \IOT. In \cite{alfuqaha-15,xu-14a}, the authors reviewed the applications, protocols and technologies used in the distinct \IOT layers, while \cite{singh-14,gubbi-13} focused on architectural aspects and \cite{tan-10,xu-14b} reviewed security ones. Most of these contributions identify a number of challenges crossing the application domain of a \DSL for \IOT, from which we identified the most relevant ones to our contribution in Section~\ref{sec:Motivation} and to which we confront our framework in Section~\ref{sec:Discussion}.

Capturing variations of a domain with explicit constructs close to the domain concepts resides at the essence of \DSLS. In that regard, many \DSLS were proposed for various purposes in the \IOT stack. \textsc{Chariot}~\cite{pradhan-15} addresses Cyber-Physical Systems by providing a component model that clearly distinguishes between communication and computation, while ensuring resilience features in highly reconfigurable systems. In~\cite{brandtzaeg-12} is presented a \DSL aimed at facilitating the deployment of applications, based on a component model of the environment used to locate the architecture nodes where business logic can be leveraged. \textsc{Alph}~\cite{munnelly-08} is a \DSL for ubiquitous healthcare that focuses on three concerns: mobility, by helping users to manage frequent devices disconnections; context-awareness to adapt application behaviour to environmental changes; and infrastructure, for managing the heterogeneity of communication protocols. Midgar~\cite{garcia-14} offers a visual interface to support end-users in controlling interconnected devices and generate the glue application making these devices interoperate. In~\cite{salihbegovic-15}, the authors present a visual \DSL for capturing the features and intercommunications of devices distributed in various application domains spanning from smart homes to patient monitoring. These contributions target different application domains at different abstraction levels, but possess every key features we identified in Section~\ref{sec:Motivation} in a more or less explicit way. Since \IOTDSL targets end-users with no prior knowledge in programming, we contrast with these contributions by offering a more intuitive, declarative style for expressing the system's dynamics through semantics rules that are compilable into a runnable \CEP engine.

ThingML~\cite{harrand-16} is the closest contribution to our \DSL: it uses a similar device description with messages and communication ports attached to devices, but describes the dynamics of devices and systems through state machines, which appear to be more obscure for end-users. However, the conceptual drawbacks are similar in both paradigms: state machines need to be deterministic on their transitions, while rules have to avoid multiple concurrent firing to avoid executing several rules at the same time. 

Other approaches, \textit{e.g.}~\cite{bhandari-13,cheng-16}, relying on the \textit{Event Condition Action} (ECA) paradigm, share a similar view for \IOT devices orchestration through \CEP, though not having the same expressiveness for devices' definition as we propose, especially with time frames and event compositions. In~\cite{shimokura-07}, the authors add \textit{pre-} and \textit{post-conditions} to ECA rules but they still do not address time frames constraints too.

All previous contributions take advantages of \MDE technologies and tools. More general \MDE framework like GeMoC~\cite{bousse-16} or ThingML allow to specialise the description of interconnected devices, for example to describe Arduino systems specifically in ArduinoML~\cite{mosser-14}. On contrary, \IOTDSL framework concentrates on generating executable rules from user-defined requirements.