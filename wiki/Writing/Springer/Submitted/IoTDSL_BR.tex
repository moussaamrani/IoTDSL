\subsection{Business Rules}
\label{sec:IoTDSL-BusinessRules}

Business rules are the core of the manipulation of \IOT systems and compose the third part of \IOTDSL as detailed in the bottom yellow part of Figure~\ref{fig:IoTDevice-MM}. This last sub-language relies on an event-based framework that allows to specify a set of \textsf{Rule}s expressing the many functionalities an end-user wants to achieve in his/her concrete configuration. 

An \IOTDSL Business Rule is identified by the keyword \textsf{rule} followed by an unique identifier, and a body of the form << \texttt{\textbf{\color{codeviolet}{when}} trigger \textbf{\color{codeviolet}{do}} reaction} >>. Rules' \texttt{trigger}s are cyclically evaluated against the surrounding environment and specify the conditions under which the corresponding \texttt{reaction}s have to be performed to realise the end-users' scenarios. A \texttt{reaction} defines actuations on the \IOT system to send or require data of identified devices, or issues events that are internally used to synchronise rules. 

%must be triggered. Rules' \textsf{trigger}s are cyclically evaluated against the surrounding environment and a \textsf{reaction} defines a sequential or parallel combination of capabilities, enabling to sort actions by, or require data from some identifiable devices. Inside the \textsf{trigger}, users typically check events to evaluate their presences, but as we will see in the following examples, they are also able to check on their absence or returned values. A typical \textsf{reaction} may be to switch on all lights in a house, or only the ones of a certain type by sending new events.

Our approach is currently purely middleware-oriented: all rules are evaluated inside a single gateway that supposedly possesses enough processing power. We leave as future work the exploration of parallelisation techniques to support multiple gateways that communicate appropriately, or the possibility to decentralise parts of the computation into nodes with sufficient processing and power resources to optimise resource consumption and lighten communication. 

We now illustrate how the scenarios Alice is concerned about (cf. Section \ref{sec:Motivation-Scenarios} can be translated into business rules in \IOTDSL with the devices' definitions detailed in Listings~\ref{lis:RE-TypeDeclarations} and \ref{lis:RE-Network}. 

\begin{description}[leftmargin=0cm]
	\item[Switching entrance lights on when coming in]  When Alice gets home (and thus opens the front door), she wants the lights to be automatically switched on in the foyer and in the living room.
	\begin{lstlisting}[language=iotdsl,
							label=lis:home-rule,
		caption=Rule to switch on the lights at home incoming]
	rule SwitchLightsWhenEntering:
		when (foyerMotion.moving after frontDoor.opened) do {
			foyerBulb.on
			livingBulb.on
		}
	\end{lstlisting}
	This rule introduces what we call \emph{facilitators}, i.e. keywords that define an unspecified time window in which a sequence of events should be observed. This time window is system-specific and needs to be defined independently in configuration files independent of descriptions in \IOTDSL. In this case, the \inlineI{foyerMotion} should detect movement \emph{nearly after} the \inlineI{frontDoor} detects an opening. 
		
	Note that reactions are defined as a sequence that does not matter: the order in which the \inlineI{foyerBulb} and the \inlineI{livingBulb} switch on largely depends on the platform capacities, i.e. they can be actuated synchronously or sequentially (in which case, no guarantee is given that the definition order will be respected). At the abstraction level \IOTDSL operates, it is irrelevant since the end user wishes to see both switched on at some point, without having to consider low-level details that would enforce such behaviour.

	\item[Illuminate bathroom when children wake up at night] When Alice's little boy wakes up at night, she would like to have the light in the bathroom to be switched on to prevent him from falling or injuring himself. Analogously, she wants the light to be switched off when he gets back to sleep afterwards.
	\begin{lstlisting}[language=iotdsl,
							label=lis:night-rule,
		caption=Rules to switch on\//off lights in the corridor at night]
		rule SwitchBathroomLightOnAtNight:	
			when (not livingLight.light() and 
					 (hallMotion.moving after childDoor.opened)) do {
				bathroomBulb.on
			}
  
		rule SwitchBathroomLightOffAtNight:	
			when (not hallMotion.movement within 3 min from childDoor.closed) do {
				bathroomBulb.off
			}
	\end{lstlisting}
	The rule \inlineI{SwitchBathroomLightOnAtNight} introduces a new keyword \inlineI{not}, which represents the \emph{absence} of a certain event type, here \inlineI{livingLight.light}. This is different than simply observing some events occuring. Note also that the second part of the rule trigger uses parenthesis to relate the facilitator \inlineI{after} to the closest event \inlineI{childDoor.open}, instead of spanning on the whole condition.

	The rule \inlineI{SwitchBathroomLightOffAtNight} presents a combination of negation with an explicit time window with the construct \inlineI{within ... from}: it indicates that no event of type \inlineI{movement} from the hall motion sensor should occur in a three-minute time window after observing the \inlineI{closed} event from the boy's door, in order to trigger the rule. \IOTDSL defines several useful time units to cope with simpler definitions (seconds, minutes, hours, or a combination of the three). 
	

	\item[Report unsupervised children on balcony] Alice considers that it is a critical situation if a child enters into the balcony without her knowledge, because of fall risks. To avoid that, she placed a switch button high enough that only an adult could press when accompanying a child outside. If the button is not press within 3 seconds after someone enters the balcony, an alarm should notify the situation.
	\begin{lstlisting}[language=iotdsl,
							label=lis:balcony-rule,
		caption=Rules to ring the alarm in case of an unsupervised child in the balcony]
	rule AlarmWhenChildOnBalcony:	
		when (not toggle.toggled within 5 sec from 
				(balconyMotion.moving after balconyDoor.opened)) do {
			alarm.sound()
		}
	\end{lstlisting}
	This last rules states that once the balcony door has been opened and movement is detected inside the balcony, the alarm should ring unless the toogle button is pressed in a five-second time window. This rule is similar to \inlineI{SwitchBathroomLightOffAtNight}, except that the baseline of the time window is here a composite event using a facilitator: once an opening followed by movement in the balcony is observed do we expect the toogle to be pressed. 
\end{description}
To summarise, an end-user uses the Business Rules sublanguage to specify the scenarios of interest in the form of \inlineI!when (trigger) do {actuations}}!: the \inlineI{trigger} condition specify the event observation pattern under which the \inlineI{actuations} are performed, by using common boolean connectors as well as time windows to observe delayed events; whereas the \inlineI{actuations} are undeterministically performed independently of their definition order. 

From a qualititive viewpoint, adopting a rule-based language presents the advantage of mimicking the cognitive process of establishing a scenario, which should ease the adoption of \IOTDSL. However, we are conscious that this requires a further examination and actual validation with end-users that are not aware of the underlying \DSL mechanisms, but we believe that presenting a visual representation for rules and powerful analysis of rule activation could ease the adoption process and facilitate scenario definitions.



